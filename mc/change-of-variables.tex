\section*{Change of Variables in Multiple Integrals}

\subsection*{Conceptual Foundations}

\begin{enumerate}

\item What is the general idea behind a change of variables in a double integral?

\item What is meant by a transformation
  \[
    T(u,v) = (x(u,v), y(u,v))?
  \]

\item What conditions must a transformation $T(u,v)$ satisfy in order for the change of variables theorem to apply?

\item What is the Jacobian matrix of a transformation $T(u,v) = (x(u,v), y(u,v))$?

\item Define the Jacobian determinant:
  \[
    J(u,v) = \frac{\partial(x,y)}{\partial(u,v)}.
  \]

\item Write the explicit formula for the Jacobian determinant in terms of partial derivatives.

\item Why does the absolute value of the Jacobian determinant appear in the change-of-variables formula?

\item What geometric meaning does the Jacobian determinant have?

\item What does it mean if the Jacobian determinant is zero at a point?

\item What happens to area elements under a transformation?

\end{enumerate}

\subsection*{Change of Variables Theorem (Double Integrals)}

\begin{enumerate}

\item State the Change of Variables Theorem for double integrals.

\item Complete the formula:
  \[
    \iint_R f(x,y)\, dA
    =
    \iint_S f(x(u,v), y(u,v))
    \left| \frac{\partial(x,y)}{\partial(u,v)} \right|
    \, du\, dv.
  \]

\item What is the relationship between the regions $R$ and $S$?

\item Why is it often easier to integrate over $S$ rather than $R$?

\item Describe the general steps required to compute a double integral using change of variables.

\end{enumerate}

\subsection*{Computing Jacobians}

\begin{enumerate}

\item Compute the Jacobian determinant of the transformation:
  \[
    x = u^2 - v^2, \quad y = 2uv.
  \]

\item Compute the Jacobian determinant for:
  \[
    x = au + bv, \quad y = cu + dv.
  \]

\item For what condition on $a,b,c,d$ is the transformation invertible?

\item How does the Jacobian determinant behave under composition of transformations?

\item If $T$ has Jacobian $J_T$ and its inverse has Jacobian $J_{T^{-1}}$, what is the relationship between them?

\end{enumerate}

\subsection*{Polar Coordinates}

\begin{enumerate}

\item Write the transformation from polar to Cartesian coordinates.

\item Compute the Jacobian determinant for the polar transformation:
  \[
    x = r\cos\theta, \quad y = r\sin\theta.
  \]

\item Why does the area element become
  \[
    dA = r\, dr\, d\theta
  \]
  in polar coordinates?

\item Rewrite the integral
  \[
    \iint_R f(x,y)\, dA
  \]
  in polar coordinates.

\item Describe a region in the plane that is particularly well suited for polar coordinates.

\item Evaluate conceptually:
  \[
    \iint_R (x^2 + y^2)\, dA
  \]
  over a disk using polar coordinates.

\end{enumerate}

\subsection*{Geometric Interpretation}

\begin{enumerate}

\item  How does a small rectangle in the $uv$-plane transform under $T$?

\item  How does the Jacobian determinant approximate local area scaling?

\item  What is the geometric interpretation of a negative Jacobian determinant?

\item Why do we take the absolute value in the change of variables formula?

\end{enumerate}

\subsection*{Strategy and Problem Solving}

\begin{enumerate}

\item  When facing a difficult region $R$, how do you decide on a good transformation?

\item  Why are linear transformations often useful for parallelogram-shaped regions?

\item  How can you transform an elliptical region into a circular region?

\item  What is the advantage of transforming complicated boundaries into rectangular ones?

\item  What common algebra mistakes occur when computing Jacobians?

\item  Why is it important to rewrite the integrand entirely in terms of $u$ and $v$?

\item  What must always be changed: the integrand, the differential, the region?

\item  How can you check whether your transformation is one-to-one?

\end{enumerate}

\subsection*{Extension to Triple Integrals}

\begin{enumerate}

\item State the change of variables formula for triple integrals.

\item Write the definition of the Jacobian determinant for a transformation
  \[
    (u,v,w) \mapsto (x,y,z).
  \]

\item How does volume scale under a transformation in $\mathbb{R}^3$?

\item What is the Jacobian determinant for spherical coordinates?

\item What is the Jacobian determinant for cylindrical coordinates?

\end{enumerate}

\subsection*{Deep Understanding / Proof-Level Questions}

\begin{enumerate}

\item How can the change of variables theorem be justified using Riemann sums?

\item Why does the determinant appear naturally in linear approximations?

\item How does the change-of-variables formula relate to linear algebra?

\item In what sense is the Jacobian determinant a multidimensional generalization of $dx/du$?

\item Why is differentiability essential in the theorem?

\item What would fail if the transformation were not invertible?

\item How does this theorem connect to the substitution rule in single-variable calculus?

\end{enumerate}
