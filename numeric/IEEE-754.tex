\section*{IEEE-754 Floating Point}

\subsection*{Basic Structure}

\begin{enumerate}
\item What are the three components of an IEEE-754 floating point number?
\item What is the purpose of the sign bit?
\item How is the exponent stored (biased or unbiased)?
\item What is the mantissa (significand), and what does it represent?
\item Write the general formula for a normalized floating point number.

\item For single precision (32-bit), how many bits are allocated to:
\begin{itemize}
\item sign
\item exponent
\item fraction
\end{itemize}

\item For double precision (64-bit), how many bits are allocated to each field?
\end{enumerate}

\subsection*{Exponent and Bias}

\begin{enumerate}
\item What is the exponent bias in single precision?
\item What is the exponent bias in double precision?
\item Why is a bias used instead of signed exponent representation?
\item How do you recover the true exponent from the stored exponent?
\item What exponent value corresponds to zero after bias correction?
\end{enumerate}

\subsection*{Normalization}

\begin{enumerate}
\item What does it mean for a floating point number to be normalized?
\item Why is there an implicit leading 1 in normalized numbers?
\item Write the formula for a normalized IEEE-754 number including the implicit bit.
\item What happens when the exponent field is all zeros?
\item What happens when the exponent field is all ones?
\end{enumerate}

\subsection*{Special Values}

\begin{enumerate}
\item How is zero represented in IEEE-754?
\item Why are there two representations of zero?
\item How is infinity represented?
\item How is NaN (Not a Number) represented?
\item What is the difference between quiet NaN and signaling NaN?
\end{enumerate}

\subsection*{Subnormal Numbers}

\begin{enumerate}
\item What are subnormal (denormal) numbers?
\item Why do subnormal numbers exist?
\item How is their representation different from normalized numbers?
\item What is the smallest positive subnormal number in single precision?
\item What is the trade-off when using subnormal numbers?
\end{enumerate}

\subsection*{Precision and Rounding}

\begin{enumerate}
\item What is machine epsilon?
\item How many significant decimal digits does single precision provide?
\item How many significant decimal digits does double precision provide?
\item What are the common IEEE-754 rounding modes?
\item What does ``round to nearest, ties to even'' mean?
\end{enumerate}

\subsection*{Range and Limits}

\begin{enumerate}
\item What is the largest representable finite number in single precision?
\item What is the smallest positive normalized number?
\item What is the smallest positive subnormal number?
\item How does exponent size affect range?
\item How does mantissa size affect precision?
\end{enumerate}

\subsection*{Conversion and Interpretation}

\begin{enumerate}
\item Convert a decimal number to IEEE-754 format step by step.
\item Convert an IEEE-754 bit pattern to decimal.
\item How do you interpret the exponent and mantissa fields together?
\item What mistakes are common when converting floating point numbers?
\end{enumerate}

\subsection*{Arithmetic and Errors}

\begin{enumerate}
\item Why is floating point addition not associative?
\item Give an example where $(a + b) + c \neq a + (b + c)$.
\item What is rounding error?
\item What is catastrophic cancellation?
\item Why do floating point errors accumulate in iterative computations?
\end{enumerate}

\subsection*{Edge Cases and Pitfalls}

\begin{enumerate}
\item Why can \texttt{0.1} not be represented exactly in binary floating point?
\item What happens when overflow occurs?
\item What happens when underflow occurs?
\item Why should you avoid direct equality comparison of floats?
\item How do you safely compare floating point numbers?
\end{enumerate}

\subsection*{Implementation Understanding}

\begin{enumerate}
\item How would you extract the sign, exponent, and mantissa from a 32-bit integer?
\item How can bit manipulation be used to inspect a float in memory?
\item What is the difference between float and double in terms of memory and precision?
\item When should you prefer double over float?
\item When might floating point be inappropriate (e.g., finance)?
\end{enumerate}
